\documentclass{jsarticle} 
\usepackage[dvipdfmx]{graphicx}
\usepackage{url}
\begin{document}

\title{ダーツの魅力}
\author{山下雄大} 
\date{2021年 4月23日}
\maketitle

\section{はじめに}
私の趣味はダーツである。私はダーツを大学1年の冬頃から友人の影響で始め、今では一人で投げに行くほどハマっている。

ダーツはお酒を飲みながらみんなでワイワイと行う印象を持たれることが多いが、突き詰めれば突き詰めるほど奥が深く、知的探究心をくすぐられる、非常に面白いスポーツである。
今回は、そんなダーツの魅力について書きたいと思う。

\section{ダーツとは}
まずはダーツがどのようなスポーツかを解説する。

\subsection{ダーツの歴史}
ダーツは今から500年以上も前に、戦場に駆り出されていたイギリス兵たちが、戦いの余暇に特定の的を目掛けて武器であった弓矢を使って競い合ったのがルーツと言われている。それから、弓を使わず矢だけを使って投げ合うようになり、的も大木を輪切りにしたものに変化し、現在の形となっていった。その後は、世界各地に広がっていき、日本にも伝わって今日のダーツブームに繋がった\cite{1}。

\subsection{ゲームについて}
ダーツには大きく分けて、カウントアップ、01(ゼロワン)、クリケットの3種類ゲームがある。ここでは、各ゲームごとに紹介して行きたいと思う\cite{2}。

\subsubsection{カウントアップ}
カウントアップは、0点からスタートして、得点を加算して行くゲームで、1ラウンド3本ずつダーツを投げていき、8ラウンド終了後の得点を競う。

カウントアップは、ダーツの中で最もシンプルなルールのゲームとなっている。

\subsubsection{01(ゼロワン)}
01の特徴は、「301,501,701,901,1101,1501」と、ゲームの種類が豊富にある。末尾が全て01であることから「ゼロワン」と呼ばれている。

01のゲームのルールは、全員同じ持ち点を持ってスタートして、(ゲームが301だった場合、301点からスタート)ヒットさせた得点が、持ち点から減っていき、持ち点をピッタリ0点にすると終了する。先に得点を0にしたプレイヤーの勝利となる。

このゲームは持ち点をピッタリ0にしなければならないので、もし持ち点を超えて得点してしまった場合 (例:持ち点残り12点で50点のブルに入れてしまった場合)は、バーストと言ってその得点は無効になり、持ち点は次のラウンドに持ち越される。

\subsubsection{クリケット}
クリケットでは15,16,17,18,19,20,ブル以外のナンバーは使わない。クリケットの基本的なルールは、有効な同一エリアに3回ヒットさせると、そのエリアは自分の陣地になる。(これをオープンという。)オープンした自分のエリアにダーツがヒットすると得点になる。また、自分がオープンしたエリアに相手がダーツを3本ヒットさせると、そのエリアは得点の入らない無効エリアとなる。(これをクローズという。)

クリケットの勝利条件は、特典が勝っている状態で全てのエリアをクローズするか、ラウンド終了時に得点の高いプレイヤーが勝利となる。

\section{ダーツの道具}
この章からは、ダーツを行うにあたって必要な道具を紹介して行きたいと思う。

\subsection{ダーツを構成している道具}
ダーツを構成している道具はチップ、バレル、シャフト、フライトの4種類がある\cite{3}。
\newpage
\subsubsection{チップ}
チップはダーツの先端に当たるパーツで、ダーツボードに刺さる部分である。比較的小さなパーツだが、メーカーによって硬さや長さ等が異なり、種類がたくさんある。

チップはダーツ全体の中で最も衝撃を受けるため、ボードに刺さって折れてしまうことも多く、ダーツをするときは常に予備のチップを持っておくことを推奨する。
\begin{figure}[h]
\begin{center}
\includegraphics[scale=0.5]{tip.png}
\caption{チップ}
\end{center}
\end{figure}

\subsubsection{バレル}
バレルは金属製のパーツで、ダーツを構成する道具の中で最も重要と言える箇所になる。基本的にはタングステンという比重の重い金属で作られている。(希少金属のため、やや高価)

バレルの種類は素材の違いや、カット(滑り止めの役割を果たす刻み)の数や間隔、バレルその物の形状や重量、重心などで多岐にわたるため、実際に投げて、自分に合ったバレルを見つけるのも楽しみの一つである。
\begin{figure}[h]
\begin{center}
\includegraphics[scale=0.5]{barrel.png}
\caption{バレル}
\end{center}
\end{figure}

\newpage
\subsubsection{シャフト}
シャフトはダーツ後部のパーツで、バレルとフライトをつなぐ役割を持っており、素材や長さ等によって全体のバランスを調整する役割も担っている。

一般的に、シャフトが長ければ弧を描くような飛びになり、シャフトが短ければ直線的に飛ぶようになる。


シャフトの種類は素材や形状や長さ等、さまざまな種類があるため、バレルと同様に、自分にあったシャフトを見つける必要がある。
\begin{figure}[h]
\begin{center}
\includegraphics[scale=0.5]{shaft.png}
\caption{シャフト}
\end{center}
\end{figure}

\subsubsection{フライト}
フライトはダーツ最後部のパーツで、投げたダーツの向きが進行方向と一致していない場合、フライトが風を受けることで起動を修正し、飛びを安定させてくれる。

一般的にフライトの面積が大きいほど飛びが安定するが、失速しやすくなる。逆に面積が小さいと軌道修正はあまり期待できないが、失速しにくく、鋭く飛ばすことができる。

フライトは特に種類が多く、形状だけでも大まかに「スタンダード」「スモール」「スリム」「ティアドロップ」「カイト」等があり、これに色や絵柄等のデザインが加わる。

フライトはダーツの道具の中でも特に面積が大きいため、プレイヤーの個性を出しやすい部分でもある。
\begin{figure}[h]
\begin{center}
\includegraphics[scale=0.5]{flight.png}
\caption{フライト}
\end{center}
\end{figure}

\newpage
\subsubsection{セッティング}
今までに紹介してきたチップ、バレル、シャフト、フライトの組み合わせのことをセッティングと言う。セッティング例を下図に示す。
\begin{figure}[h]
\begin{center}
\includegraphics[scale=0.3]{setting.png}
\caption{セッティング}
\end{center}
\end{figure}

\section{レーティング}
ダーツにはレーティングというシステムがある。これは、自分の実力を知る上で重要なシステムである。この章では、レーティングについて紹介して行きたいと思う。

\subsection{レーティングとは}
レーティングとは自分のダーツの実力を示す指標のことで、自分がどれくらいの実力なのかを把握することで、対戦相手の選定やレベルアップするための課題が明確になり、楽しく、無理のないステップアップにつながる\cite{4}。

\subsection{レーティングの出し方}
レーティングは、対人戦で01ゲームとスタンダードクリケット(どちらかでも可能)を行なった際のスタッツによって算出される。

スタッツとは、ゲーム終了時に出される、そのゲームの成績(平均点数もしくは平均本数)のこと。この数字から自分のアベレージ(平均成績)がわかり、そのアベレージからレーティングを出す。
\newpage
\subsection{レーティングとフライト}
ダーツにはレーティングのほかに、フライト呼ばれる階級制度が存在する。この制度により、自分がどの程度の実力を持っているのかが一目でわかる。そして、これから目指すべき目標の設定もしやすくなっている。以下に、レーティングのシステムをイメージしやすいように、レーティング表を示す。
\begin{table}[h]
\begin{center}
\begin{tabular}{|c|c|c|c|}
    RATING & 01 STATS & CRICKET STATS & FLIGHT \\
    1 & 0.0〜 & 0.0〜 & C \\
    2 & 40.0〜 & 1.30〜 & C \\
    3 & 45.0〜 & 1.50〜 & C \\
    4 & 50.0〜 & 1.70〜 & CC \\
    5 & 55.0〜 & 1.90〜 & CC \\
    6 & 60.0〜 & 2.10〜 & B \\
    7 & 65.0〜 & 2.30〜 & B \\
    8 & 70.0〜 & 2.50〜 & BB \\
    9 & 75.0〜 & 2.70〜 & BB \\
    10 & 80.0〜 & 2.90〜 & A \\
    11 & 85.0〜 & 3.10〜 & A \\
    12 & 90.0〜 & 3.30〜 & A \\
    13 & 95.0〜 & 3.50〜 & AA \\
    14 & 102.0〜 & 3.75〜 & AA \\
    15 & 109.0〜 & 4.00〜 & AA \\
    16 & 116.0〜 & 4.25〜 & SA \\
    17 & 123.0〜 & 4.50〜 & SA \\
    18 & 130.0〜 & 4.75〜 & SA
\end{tabular}
\end{center}
\end{table}

\section{まとめ}
今回はダーツの魅力を知ってもらうために、ダーツの歴史やゲーム、ダーツを構成する道具や、レーティングシステムについて紹介してきた。これを読んで少しでもダーツについて興味を持ってもらえたら幸いである。

\newpage
\begin{thebibliography}{9}

\bibitem{1}
エスダーツ"ダーツとは".エスダーツホームページ. \url{https://www.s-darts.com/project/darts_kouza/about.html},2021年4月23日アクセス

\bibitem{2}
DARTSLIVE."はじめてのダーツ|ゲームのルール".DARTSLIVEホームページ. \url{https://www.dartslive.com/jp/beginner/},2021年4月23日アクセス

\bibitem{3}
ダーツノミカタ."ゼロから学ぶ初心者ガイド!ダーツの道具・仕組みを知ろう!". \url{https://www.dartshive.jp/html/page64.html},2021年4月23日アクセス

\bibitem{4}
ダーツノミカタ."自分の腕前を示す指標、レーティングを知ることで更なるレベルアップに!". \url{https://www.dartshive.jp/html/page47.html},2021年4月23日アクセス

\end{thebibliography}

\end{document}